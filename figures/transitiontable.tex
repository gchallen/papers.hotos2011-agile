\begin{table*}[t]
\begin{tabularx}{\textwidth}{p{3in}p{3in}}
{\small
\begin{tabularx}{\columnwidth}{cX}

\textbf{t} &
\multicolumn{1}{c}{\textbf{Description}}
\\ \toprule 

0 & With no tasks to process, \textbf{P1} and \textbf{M1} are idle, while
\textbf{R1} operates at low duty cycle awaiting incoming tasks.
\\

1 & 
Alerted to an incoming task, the device activates \textbf{R2} to rapidly
receive task data and \textbf{S1} to store it.
\\

2 &
When processing begins, energy usage is reconfigured by activating
\textbf{P2} and disabling \textbf{R2}.
\\

3 &
Responding to a spike in availability, task processing is accelerated by
activating \textbf{M2} and mirroring to \textbf{S2}.
\\

\end{tabularx}
}

&

{\small
\begin{tabularx}{\columnwidth}{cX}

\textbf{t} &
\multicolumn{1}{c}{\textbf{Description}}
\\ \toprule 

4 &
As results start being written to disk, power usage shifts within the same
component ensemble.
\\

5 & 
When power availability suddenly drops to 1~W, the device shifts to storing
all results on \textbf{S1}, and compresses data on the fly, resulting in
higher usage of \textbf{P2}.
\\

6 &
To return results, \textbf{R3} is activated and driven using \textbf{P1}.
\\

7 &
When power availability falls again, the device disables \textbf{M3} and
switches to the smaller memory chip \textbf{M1}.
\\

8 &
Processing is complete.
\\

\end{tabularx}
}
\end{tabularx}
\caption{\small \textbf{Test.} Test}
\label{table-transitions}
\vspace{0.10in}
\hrule
\vspace{-0.20in}
\end{table*}
