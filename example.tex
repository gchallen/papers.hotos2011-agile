\section{Architecture}
\label{section-architecture}

\begin{table}[t]
\begin{threeparttable}
{\footnotesize
\begin{tabularx}{\columnwidth}{p{0.05in}lr@{.}lX}

\textbf{ID} &
\multicolumn{1}{c}{\textbf{Name}} &
\multicolumn{2}{c}{\textbf{mW}} &
\multicolumn{1}{c}{\textbf{Performance}}
\\ \toprule 

\multirow{2}{*}{\textbf{P1}} & \multirow{2}{*}{ARM Cortex-M4\tnote{1}~\cite{cortexm4-web}} &
0 & 9\tnote{2} &
75 MHz
\\

& &
15 & 6 &
300 MHz
\\ \cmidrule(lr){2-5}

\multirow{2}{*}{\textbf{P2}} & \multirow{2}{*}{ARM Cortex-A9~\cite{cortexa9-web}} &
23 & 5\tnote{2} &
415 MHz
\\

& &
400 & &
830 MHz
\\ \cmidrule(lr){1-5}

\multirow{2}{*}{\textbf{M1}} & \multirow{2}{*}{\textbf{32~MB} ISSI SDRAM~\cite{issi32MBsdram-datasheet}} &
81 & &
Refresh only
\\

& &
108 & &
166 MHz
\\ \cmidrule(lr){2-5}

\multirow{2}{*}{\textbf{M2}} & \multirow{2}{*}{\textbf{1~GB} Micron ``Slow'' DDR2} &
322 & \tnote{4} &
Refresh only
\\

& &
482 & &
266 MHz
\\ \cmidrule(lr){1-5}

\multirow{2}{*}{\textbf{S1}} & \multirow{2}{*}{\textbf{2~GB} MicroSD Card} &
20 & \tnote{5} &
Idle
\\

& &
100 & \tnote{5} &
25 MBps\tnote{6}
\\ \cmidrule(lr){2-5}

\multirow{2}{*}{\textbf{S2}} & \multirow{2}{*}{\textbf{64~GB} OCZ SSD} &
200 & \tnote{7} &
Idle
\\

& &
1000 & \tnote{7} &
5.5 MBps\tnote{7}
\\ \cmidrule(lr){1-5}

\multirow{2}{*}{\textbf{R1}} & \multirow{2}{*}{\textbf{250~kbps} TI CC2540 BLE} & 
6 & 7\tnote{3} &
10\% duty cycle\tnote{8}
\\

& &
66 & 3\tnote{9} &
Receive mode\tnote{9}
\\ \cmidrule(lr){2-5}

\multirow{2}{*}{\textbf{R2}} & \multirow{2}{*}{\textbf{11~Mbps} Marvell 802.11bg} &
30 & 9\tnote{3} &
10\% duty cycle\tnote{8}
\\

& &
309 & 3\tnote{10} &
Idle mode\tnote{10}
\\

\end{tabularx}
}
{\footnotesize
\begin{tablenotes}
\item [1] Capable of running a subset of the full P2 ISA.
\item [2] Optimistic estimate based on an optimistic estimate of DVFS providing 1:5 performance and
1:17 power scaling~\cite{jssc02-PowerPC-SoC}.
\item [3] Estimated based on scaled full-power performance.
\item [4] Estimated based on Micron leakage numbers.
\item [5] Estimated due to lack of publicly-available datasheets.
\item [6] Maximum achievable.
\item [7] Measured by Tom's Hardware~\cite{ssd-tomshardware}.
\item [8] Duty cycling shifts power usage from the receiver to the sender,
which has to remain online (as in 802.11 PSM) or send longer packets (as in
802.15.4 Low-Power Listening~\cite{tinyos-lpl}).
\item [9] Receive-only in high-sensitivity mode. Transmit is similar.
\item [10] Transmit and receive vary so usage is workload-dependent.
\end{tablenotes}
}
\vspace*{-0.05in}
\caption{\small \textbf{Performance and power consumption of selected hardware
components.} We assume voltage gating can reduce the usage of disabled
components to near zero~\cite{islped-vdd-gate}.}
\end{threeparttable}
\label{table-components}
\vspace{0.10in}
\hrule
\vspace{-0.20in}
\end{table}


To begin, we assemble a power-proportional hardware architecture suitable for
a mobile device by choosing two components in each class---processors,
memory chips, storage devices and radios. While some components such as
processors can scale performance and power this is usually over a limited
range. Due to the realities of hardware design, the power-performance
relationship for a given component is the result of a fixed set of design
choices firmly embedded in the component itself. So rather than focusing on
power-performance scaling within single chips, future power-proportional
architectures will achieve scaling by incorporating multiple components, each
operating efficiently within a different power-performance range.

Table~\ref{table-components} summarizes properties of the eight components we
chose. The relationship between power and performance is complex for each
component. Processors provide a smooth performance transition over a
restricted power envelope using DVFS, but cannot scale to zero due to leakage
current that increases as a function of the number of transistors. Memory
chips have a fixed refresh cost that scales somewhat with their size and an
addition scaling component corresponding to usage. Storage devices differ
based on whether they include spinning components. Flash drives do not and
hence scale power roughly with usage but are presently limited in size.
Radios exhibit even more interesting power-performance variation because
their power usage depends both on the hardware and the protocol. 802.11
clients can enter power-saving mode (PSM) which leverages base station
infrastructure to save client power. Bluetooth has limited range but lower
power consumption balanced between both ends of the link.

Using the components we selected allows our device to operate across a wide
power range. We refer to a \textit{component ensemble} as the set of
components active at a given time, constraining the possible ensembles to
consist of (a) at least one or multiple processors and memory chips and (b)
zero, one or multiple storage devices or radios\footnote{While many low-power
processors come with small amounts of integrated memory, we have
conservatively chosen to require 32~MB of RAM in order to run embedded
versions of Linux. It is conceivable that our candidate device could enter an
active sleep state with a micro-kernel capable of fitting in the processor's
onboard RAM.}. It its lowest-power ensemble, the device has a 75~MHz CPU and
32~MB of RAM, draws as little as 82~mW\footnote{Actual power consumption
would be higher due to system buses, memory controllers, and other components
of a complete architecture.} and is roughly-equivalent to a embedded sensor
node; in its highest-power ensemble with all components powered and active
the device has multiple cores, over 1~GB of RAM, over 320~GB of storage, Wifi
and Bluetooth. Consuming almost 2.5~W, it is similar to emerging smartphones.

\begin{figure*}[t]
\includegraphics{./figures/componentgraph.pdf}

\caption{\small \textbf{Power envelopes of all 144 2x2x2x2 device component
ensembles.} Ensembles are sorted by increasing maximum power draw. For each
ensemble, the bottom shows which components are active and the top displays
the power envelope. The top 80\% of the envelope---the efficient operating
range---is drawn in dark blue. The right axis counts the total number of
ensembles that might draw that much power: e.g., there are
121 ensembles that could consume 0.75~W, depending on the workload.}
\vspace{0.10in}
\hrule
\vspace{-0.20in}
\label{figure-componentgraph}
\end{figure*}


In total our device can activate \textit{144 possible component ensembles}.
Figure~\ref{figure-componentgraph} shows all possibilities and the power
range associated with each, and illustrates several key observations. First,
there is wide variation in the power usage of component ensembles even in an
architecture with only two components per class. Incorporating more
components per class would result in even more variation. Second, at any
overall power level there are many components ensembles that the device can
use. A diverse set of ensembles is available at most power points: a fast
processor, small memory chip, and slow disk; a slow processor, large memory,
and a fast radio; etc. Of course, not all possible ensembles at a given power
point will be possible or sensible---a processor of sufficient speed may be
required to drive a high-bandwidth radio---but, given careful component
choices, many will. It also may not make sense to operate a component
ensemble at a power level insufficiently to allow high levels of component
activity, but given the variations in load and availability and the potential
overhead of transitions between component ensembles we believe that devices
will spend some time at the low end of ensemble power envelopes.

