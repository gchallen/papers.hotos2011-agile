\section{Introduction}

Battery-powered devices are trapped by trends. Better performance requires more
power, battery technologies are improving
slowly~\cite{economist-batteryhistory}, and users want their devices to last as
long as they to today---or longer~\cite{informationweek-batteries}.

A way to escape this trap leverages power-proportional hardware
architectures~\cite{barroso-energyproportional} that deliver good performance
when needed and consume little power when idle. Because most device
components are tuned to operate efficiently within a narrow power-performance
range, we expect future power-proportional architectures to be increasingly
\textit{heterogeneous}: featuring multiple different processors, memory
banks, storage devices and radios, each component embodying a particular
power-performance tradeoff. Heterogeneity will produce a single device that
can morph into many others: phones that can sprint like a laptop and sleep
like a sensor node.

Indeed, many devices already incorporate multiple processors, storage devices
and radios with different power-performance characteristics. The ACPI
specification~\cite{acpi-standard} standardizes device and global power
states. Researchers have proposed operating system designs acknowledging this
heterogeneity~\cite{baumann-barrelfish}, performance- or power-driven
component combinations~\cite{mogul-hybridnvmdram,aruna-3Gwifi}, approaches
that harness the power efficiency of a particular set of components to a
particular task~\cite{andersen-fawn,szalay-amdahl}, and systems organized
into multiple power-performance tiers~\cite{sorber-turducken}. Inspired by
these efforts, we introduce the term \textit{power agility} to describe the
ability of a system to operate a heterogeneous power-proportional device
balancing performance and power consumption.

Given increasingly heterogeneous devices, power agility requires not merely
adjusting individual components but activating and deactivating them to react
to changes in demand caused by variations in device usage. The idle phone in
my pocket consumes less power than the one routing me to my destination, and
while the mapping application wants the high-power radio, the game prefers a
faster processor. So while power-proportional \textit{hardware} allows the
device to sprint and sleep, power-agile \textit{software} guides it nimbly
between states in response to demand. We consider hard attempts to mask power
fluctuations~\cite{rangan-hpca11} a mistake. Only the operating system has
enough information to achieve power agility.

This paper outlines the principles of power-agile computing. First, we
construct a heterogeneous power-proportional device illustrating the size and
diversity of the state space inherent to these architectures. Next, we
present a scenario demonstrating a power-agile responding to changes in
demand. Using this scenario, we develop a set of challenges inherent to
power-agile operation and discuss approaches to overcoming them.
