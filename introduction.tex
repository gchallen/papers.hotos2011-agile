\section{Introduction}

A growing class of battery-powered consumer devices are trapped between two
trends: users' desire for better performance, and the slow pace of
improvements in battery technology. Users want more from their phones: more
apps, more access to data, more performance. But they don't want these things
at the expense of traditional phone battery lifetimes or form factors.

Given the very gradual improvements in battery density over the last century,
we expect that mobile devices will attempt to slip this trap through
increasingly power-proportional hardware architectures. And because desiging
single components that can operate efficiently over wide power-performance
ranges is difficult, we expect that these power-proportional architectures
will look increasingly heterogeneous, featuring \textit{multiple} processors,
memory banks, storage devices and radios. Indeed, today's smartphones already
feature multiple storage devices and radios with different power-performance
characteristics, and rumors have been circulating about multi-processor
features in the next iPhone4. Emerging power-proportional heterogeneous will
produce devices that can occupy many different states: phones that can both
sprint like a laptop and sleep like a wireless sensor node.

But these abilities will not be effectively utilized without system support.
This paper introduces the term \textit{power agility} to describe the ability
of a system to efficiently operate power proportional hardware to balance
performance and power consumption. Given increasingly heterogeneous devices,
power agility requires not merely adjusting component parameters such as DVFS
but activating and deactivating components in reaction to changing demand and
availability. Energy demand changes as devices are used: the phone idle in my
power requires much less power than when it is actively interacting with the
user. And energy availability changes as well, either because power is
rationed to improve device lifetime or when power is harvested from the
environment: the wind-powered embedded micro-server has a lot of available
energy when a breeze is up, and much less when it is still. So while
power-proportional hardware allows the cat to sprint and sleep, it's
power-agile software that lets it awaken instantly, spring into action, and
maneuver deftly across the room before landing in your lap, purring.

This paper outlines the principles of power-agile computing necessary to
efficiently operate tomorrow's heterogeneous power-proportional hardware
architectures. We begin by constructing a power-proportional device using
multiple components to illustrate the diversity and flexibility of the state
space provided by these architectures. Next, we walk through an example
scenario demonstrating how a power-agile system transitions between hardware
states as need and availability change. Finally, we use this scenario to
develop a set principles underlying power-agile operation --- balance,
coordination, speed, and reflexes --- and discuss approaches to designing
systems with these properties.
