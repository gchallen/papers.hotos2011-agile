\section{Introduction}

Battery-powered devices are trapped by trends. More powerful performance
requires more power, and while battery technologies slowly
improve~\cite{economist-batteryhistory} users want more capable devices that
last as long as they do today or longer~\cite{informationweek-batteries}. A
way to escape this trap leverages power-proportional hardware
architectures~\cite{barroso-energyproportional} that scale power consumption
to perform when needed and draw little power when idle. Because most device
components are tuned to operate efficiently within a narrow power-performance
range, we expect future power-proportional architectures to be increasingly
\textit{heterogeneous}, featuring multiple different processors, memory
chips, storage devices and radios, each with different power-performance
tradeoffs. Heterogeneity produces devices with fluid characteristics: phones
that sprint like desktops and sleep like sensor nodes.

Today many devices already incorporate multiple processors, storage devices
and radios with different power-performance characteristics. Researchers have
proposed operating system designs acknowledging this
heterogeneity~\cite{baumann-barrelfish}, performance- or power-driven
component combinations~\cite{mogul-hybridnvmdram,aruna-3Gwifi}, approaches
harnessing the efficiency of a particular set of components for a certain
task~\cite{andersen-fawn,szalay-amdahl}, and systems organized into multiple
power-performance tiers~\cite{sorber-turducken}. Inspired by these efforts,
we introduce the term \textit{power agility} to describe the ability of a
system to operate a heterogeneous power-proportional device balancing
performance and power consumption.

Given increasingly heterogeneous devices, power agility requires not merely
adjusting individual components but activating and deactivating them to react
to changing demand. The idle phone in my pocket consumes less power than the
one routing me to my destination, and while the mapping application wants the
high-power radio, the game prefers a faster processor. So while
power-proportional \textit{hardware} allows the device to sprint and sleep,
power-agile \textit{software} guides it correctly between states. Recent
microarchitectural advances attempt to mask hardware heterogeneity from the
operating system~\cite{rangan-hpca11}, but we consider these a mistake. Only
the operating system has the system-wide visibility and application
information to achieve power agility.

This paper outlines the principles of power-agile computing. To begin, we
design a heterogeneous power-proportional device to illustrate the size and
diversity of the state space inherent to these architectures. Next, we
present a scenario demonstrating our device responding to changes in demand.
Using this scenario, we develop a set of challenges inherent to power-agile
operation and discuss approaches to overcoming them.
