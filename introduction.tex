\section{Introduction}

Battery-powered devices are trapped by trends. Better performance requires
more power, battery technologies are improving
slowly~\cite{economist-batteryhistory}, and users want their devices to last
as long as they to today---or longer~\cite{FIXME}.

The way forward utilizes power-proportional hardware
architectures~\cite{barroso-energyproportional} to deliver good performance
when needed and consume little power when idle. Because most device
components are tuned to operate efficiently within a narrow power-performance
range, we expect future power-proportional architectures to be increasingly
\textit{heterogeneous}: featuring multiple different processors, memory
banks, storage devices and radios, each component embodying a particular
power-performance tradeoff. Heterogeneity will produce a single device that
can morph into many others: phones that can compute like a laptop and sleep
like a sensor node.

Indeed, many devices already incorporate multiple processors, storage devices
and radios with different power-performance characteristics. Recent research
efforts have proposed operating system designs acknowledging this
heterogeneity~\cite{FIXME-Barrelfish}, performance- or power-driven component
combinations~\cite{FIXME-HybridDRAM,FIXME-ArunaWork}, approaches that harness
the power efficiency of a particular set of components to a particular
task~\cite{FIXME-FAWN,FIXME-Amdahl}, and organizing systems into multiple
power-performance tiers~\cite{FIXME-Turducken}. Inspired by these efforts, we
introduce the term \textit{power agility} to describe the ability of a system
to operate a heterogeneous power-proportional device balancing performance
and power consumption. Power agility is what allows the system to answer the
question: of all the devices I \textit{could} be, which one \textit{should} I
be?

Given increasingly heterogeneous devices, power agility requires not merely
tuning per-component parameters such as DVFS but activating and deactivating
entire components to react to changes in demand caused by variations in
device usage. The idle phone in my pocket consumes less power than the one
routing me to my destination, and while the mapping application benefits from
a high-power radio, the game prefers a faster processor. So while
power-proportional hardware allows the cat to sprint and sleep, it's
power-agile software that lets it awaken instantly, spring into action, and
maneuver deftly across the room before landing in your lap, purring.

This paper outlines the principles of power-agile computing necessary to
efficiently operate tomorrow's heterogeneous power-proportional hardware
architectures. We begin by constructing a power-proportional device using
multiple components to illustrate the diversity and flexibility of the state
space provided by these architectures. Next, we walk through an example
scenario demonstrating how a power-agile system transitions between hardware
states as demand changes. Using this scenario, we develop a set of challenges
inherent to power-agile operation and discuss approaches to overcoming them.
