\section{Scenario}
\label{section-scenario}

\begin{figure*}
\noindent\begin{minipage}[t]{0.5\textwidth}
\vspace{0pt}
{\small
\begin{tabularx}{\columnwidth}{cX}

\textbf{t} &
\multicolumn{1}{c}{\textbf{Description}}
\\ \toprule 

0 & With no tasks to process, \textbf{P1} and \textbf{M1} are idle, while
\textbf{R1} operates at low duty cycle awaiting incoming tasks.
\\

1 & 
Alerted to an incoming task, the device activates \textbf{R2} to rapidly
receive task data and \textbf{S1} to store it.
\\

2 &
When processing begins, energy usage is reconfigured by activating
\textbf{P2} and disabling \textbf{R2}.
\\

3 &
Responding to a spike in availability, task processing is accelerated by
activating \textbf{M2} and mirroring to \textbf{S2}.
\\

\end{tabularx}
}
\end{minipage}
\begin{minipage}[t]{0.5\textwidth}
\vspace{0pt}
{\small
\begin{tabularx}{\columnwidth}{cX}

& \\ \toprule 

4 &
As results start being written to disk, power usage shifts within the same
component ensemble.
\\

5 & 
When availability drops to 1~W, the device stores compressed results on
\textbf{S1}.
\\

6 &
To return results, \textbf{R3} is activated and driven using \textbf{P1}.
\\

7 &
When power availability falls again, the device disables \textbf{M3} and
switches to the smaller memory chip \textbf{M1}.
\\

8 &
Processing is complete.
\\

\end{tabularx}
}
\end{minipage}
\caption{\small \textbf{Test.} Test}
\label{table-transitions}
\vspace{0.10in}
\hrule
\vspace{-0.20in}
\end{figure*}


We illustrate how a power-agile device might operate using a scenario. To
incorporate both changes in power demand and availability, we imagine a
wind-powered microserver receiving, processing, storing and offloading a
task, while also experiencing fluctuations in power availability due to
harvesting variation. Figure~\ref{figure-transitiongraph} describes the
scenario in detail and illustrates our power-agile architecture transitioning
through several component ensembles as demand and availability change. It
shows how the per-component power allocation changes as the task moves
through different stages, and how overall and per-component power allocations
change to match power availability. Examining this scenario produces several
observations which underpin the principles of power-agile operation presented
in the next section:

\begin{enumerate}

\item Ensemble transitions can be the result of changing availability (t = 3,
t = 7) or demand (t = 2, t = 6). Thus, \uline{power-agility is crucial even
for systems designed to operate on a static or slowly-changing power budget},
such as cell-phones, as well as for devices operating off of harvested
energy.

\item \uline{Energy usage may shift even within the same component ensemble.}
At t = 4 moving from task processing to storing results reconfigures the
distribution of power usage within the same component ensemble. This is a
result of the dependency of component power consumption on usage and
complicates the process of performing ensemble transitions, as the precise
power draw of a given task in a different ensemble may be difficult to
predict.

\item \uline{Power-agile operating systems should anticipate ensemble
transitions}, exploiting component heterogeneity to shift power towards
components most useful to complete the task. At t = 2, as the task begins
processing data the system shifts power from the radio to the CPU as demand
shifts from communication to computation.

\item \uline{Power-agile operating systems must prepare for ensemble
transitions.} This is particularly important when power availability is
changing or when the system anticipates that the devices is transitioning
between states with different power requirements. Depending on the layout and
status of pages in M2 at t = 7, when power availability changes,
transitioning from M2 to M1 may be easy or it may be difficult. And at t = 6,
where demand requirements change, the system should be prepared to migrate
threads from P2 to P1.

\end{enumerate}
